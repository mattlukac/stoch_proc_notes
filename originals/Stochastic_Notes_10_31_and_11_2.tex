\documentclass{article}

\usepackage{amsmath, amssymb}

\usepackage{amsthm}

\usepackage{pgf,tikz,pgfplots}
\pgfplotsset{compat=1.15}
\usepackage{mathrsfs}
\usetikzlibrary{arrows}


\theoremstyle{definition}
\newtheorem{lem}{Lemma}
\newtheorem{thm}{Theorem}
\newtheorem{fact}{Fact}
\newtheorem{cor}{Corollary}
\newtheorem{prop}{Proposition}

\DeclareMathOperator{\e}{\mathbb{E}}
\DeclareMathOperator{\p}{\mathbb{P}}
\DeclareMathOperator{\var}{Var}

\newcommand{\lr}[1]{\left(#1\right)}
\newcommand{\lrb}[1]{\left[#1\right]}

\newcommand{\R}{\mathbb{R}}


\begin{document}
\section*{10/31}
\section*{Cell Proliferation and Cancer}
The following setup is the motivation for a later theorem on Hitting Probabilities. Mutations can affect the rates of cell proliferation (splitting) and death. Suppose we counted the number of occurences of a particular mutation in each of many tissue samples. We wonder if the distribution of mutation numbers is consistent with a \underline{driver mutation}, i.e. one with a higher than normal rate of proliferation\footnote{Here, normal would mean that the rate of cell proliferation equals the rate of cell death.}.


\section*{Model}
Let $X_t$ be the number of mutated cells at time $t$, so $X_t \in \mathbb{N}_{\geq 0}$, and assume that $X_0 =1$ (so we start with one mutated cell). 

Assume that each cell divides at rate $\lambda$ and dies at rate $\mu$. This means that
\begin{align*}
	X_t &\to X_t + 1 \quad \text{at rate $\lambda X_t$}\\
	X_t & \to X_t - 1 \quad \text{at rate $\mu X_t$}.
\end{align*}

Questions:
\begin{enumerate}
	\item What is the probability that the mutation ``grows?" That is, for some large $N$, what is 
	\[
		\p\{X_t = N \text{ before } X_t \text{ hits } 0 \}= ?
	\]
	\item If the mutation is not a driver but present due to an ongoing mutation of rate $a$, what is the expected distribution of the number of mutations per tissue sample?
\end{enumerate}


\section*{Hitting Probabilities} To answer question 1 we need a bit more theory. Suppose $(X_t)_{t\geq 0}$ is a Continuous Time Markov Chain with generator $G$ on a state space $\mathcal{X}$. For any subset $A \subset \mathcal{X}$, we define
\[
	\tau_A = \inf\{t\geq 0 \mid X_t \in A\},
\]
which is the first time that the Markov chain lands in $A$. If $A=\{x\}$ we just write $\tau_x$, which is the first time the Markov chain hits $x$. 
We now state the theorem:
\begin{thm}[Harmonic Functions] Suppose that $A,B\subset X$ are disjoint subsets with the property that 
\[
	\p\{ \tau_{A\cup B} <\infty\} = 1.
\]
For $x\in \mathcal{X}$ we define 
\[
	h(x) = \p^x \{\tau_A <\tau_B\} := \p\{\tau_A<\tau_B \mid X_0 = x\}.
\]
Then the function $h$ is the unique function satisfying the following two conditions:
\begin{enumerate}
	\item $
		h(x) = \begin{cases}
			1 & \text{ for } x \in A\\
			0 & \text{ for} x\in B
		\end{cases}
	$
	\item $Gh(x) = 0$ for $x\notin A\cup B$. 
\end{enumerate}
\end{thm}

To prove the theorem we need two lemmas.
\begin{lem}
Let $T_1,\dots, T_n$ be independent random variables with $T_i\sim \text{Exp}(\lambda_k)$. Let $T := \min\{T_1,\dots,T_n\}$. Then $T\sim \text{Exp}\lr{\sum_{k=1}^n \lambda_k}$ and 
\[
	\p\{T_k = T\} = \frac{\lambda_k}{\lambda_1+\dots+\lambda_n}.
\]
\end{lem}

\begin{proof}[Proof of Lemma 1]
	\begin{align*}
		\p\{T >t \} &= \p\{T_k > t \quad \forall k\}\\
		&= \prod_{k=1}^n \p\{T_k > t\} &&\text{(by independence)}\\
		&= \prod_{k=1}^n e^{-\lambda_k t} \\
		&= e^{-t \sum_{k}\lambda_k}
	\end{align*}
	which establishes the first claim. For the second claim, first note that
	\begin{align*}
		\p\{ T_k \in dt, T_j > t \quad \text{for } j\neq k\} &= \lambda_k e^{-\lambda_k t} \prod_{j\neq k} e^{-\lambda_j t}\\
		&= \lambda_k e^{-t \sum_j \lambda_j},
	\end{align*}
	so we now integrate over $t$ to get
	\[
		\p\{T_k = T\} = \lambda_k \int e^{-t \sum_j \lambda_j}\,dt = \frac{\lambda_k}{\lambda_1+\dots + \lambda_n}.
	\]
\end{proof}

\begin{lem} 
Let $\tau_+ = \inf\{ t\geq 0 \mid X_t \neq X_0\}$ be the time of the first jump. Then, if $X_{0} = x$, then $\tau_+ \sim \text{Exp}(-G_{xx})$.

\end{lem}



\section*{11/2}

\begin{lem}
	Let $\tau_+ = \inf\{ t\geq 0 \mid X_t \neq X_0\}$ be the time of the first jump. Suppose $\p\{\tau_+ > 0\} >0$. Then $\tau_+ \sim \text{Exp}(\lambda)$ for some $\lambda>0$. 
\end{lem}
\begin{proof}
	Let $F(t) = \p\{\tau_+ > t\}$ be the distribution function for $\tau_+$. By the Markov property on $X_t$, we have
	\[
		\p\{\tau_+ > t+s \mid \tau_+ > t\} = \p\{\tau_+ > s\} = F(s).
	\]
	Unraveling the conditional probability on the left, we find 
	\[
	F(s) = \frac{\p\{\tau_+>t+s \text{ and } \tau_+ > t\}}{\p\{\tau_+ > t\}} = \frac{F(t+s)}{F(t)}.
	\]
	We then have that $F(t+s) = F(t) F(s)$, and the only continuous solution to this equation is $F(t) = e^{-\lambda t}$ for some $\lambda >0$.
	
\end{proof}

To solve our problem with splitting and dying cells, we model it with a continuous time Markov chain with state space $\{1, 2, \dots, N\}$ and generator $G$ given by
\begin{align*}
	G = \begin{pmatrix}
		0 & 0 & 0 & 0 & 0 &\dots & 0\\
		\mu & -\mu-\lambda & \lambda & 0 & 0 & \dots & 0\\
		0 & 2\mu & -2\mu-2\lambda & 2\lambda & 0 & \dots & 0\\
		0 & 0 & 3\mu & -3\mu-3\lambda & 3\lambda & \dots & 0 \\
		\vdots & & & & & & \vdots 
	\end{pmatrix}
\end{align*}

Let $h(x) = \p^x\{\tau_N<\tau_0\}$ be the hitting probability. From the theorem on hitting probabilities, we know that 
\begin{align*}
	h(0) &= 0 \\
	h(N) &= 1\\
	Gh(x) &= 0 \quad \forall x\neq 0, N.
\end{align*}
The last of these equations says
\[
	x\lambda (h(x+1) - h(x)) + x \mu( h(x-1)-h(x)) = 0
\]
which rearranges into
\[
	h(x+1) - h(x) = \frac{\mu}{\lambda} (h(x) - h(x-1)).
\]
Since $h(1) - h(0) = h(1)$ by our boundary condition we can use this as the seed for this recurrence relation. We get
\[
	h(x+1) - h(x) = \left(\frac{\mu}{\lambda}\right)^x h(1),
\]
so
\begin{align*}
	h(x) &= h(1) + \sum_{y=1}^{x-1} h(y+1) - h(y)\\
	&= h(1)\left\{1 + \sum_{y=1}^{x-1} \left(\frac{\mu}{\lambda}\right)^y\right\}\\
	&= h(1) \left(\frac{1- \left(\frac{\mu}{\lambda}\right)^x}{1-\frac{\mu}{\lambda}}\right).
\end{align*}
Next we use the other boundary condition:
\[
	1 = h(N) = h(1)\left(\frac{1- \left(\frac{\mu}{\lambda}\right)^N}{1-\frac{\mu}{\lambda}}\right)
\]
which gives
\[
	h(1) =  \left\{\frac{1- \left(\frac{\mu}{\lambda}\right)^N}{1-\frac{\mu}{\lambda}}\right\}^{-1},
\]
and
\[
	h(x) = \frac{1- \left(\frac{\mu}{\lambda}\right)^x}{1 - \left(\frac{\mu}{\lambda}\right)^N} .
\]

This gives us the answer to our first question: what is the probability that a single mutation ``grows?" This is answered by the value 
\[
	h(1) =  \frac{1- \left(\frac{\mu}{\lambda}\right)}{1 - \left(\frac{\mu}{\lambda}\right)^N} \approx 1-\frac{\mu}{\lambda} \text{ for large }N.
\]

\section*{Stationary Distributions} 

\begin{prop}
Let $(X_t)_{t\geq 0}$ be a continuous time Markov chain with generator $G$ on state space $\mathcal{X}$. Suppose that $G^T \pi = 0$ for some probability distribution $\pi$ on $\mathcal{X}$. Then
\[
	\p^\pi\{X_t = y\} = \pi(y).
\]
\end{prop}

The conditions on $\pi$ above mean
\[
	\sum_{y\in \mathcal{X}} \pi(y) = 1, \quad \pi(y)\geq 0, \quad \sum_{y\in \mathcal{X}} \pi(y) G_{yx} = 0.
\]
The notation $\p^\pi$ means
\[
	\p^\pi\{X_t=y\} = \sum_{x\in\mathcal{X}} \pi(x) \p^x\{X_t=y\}.
\]

Such a distribution $\pi$ is called a \emph{stationary distribution}.


\begin{proof}
Let $f:\mathcal{X}\to \R$ be a function (thought of as a column vector) with $\sum_y \pi(y)f(y) < \infty$ (if we wished to generalize to a countable state space). Recall that
\[
	\e^x[f(X_t)] = P_t f(x) = e^{tG}f(x).
\]
Since $\pi G = 0$, 
\[
	\pi P_t = \pi e^{tG} = \pi \sum_{n\geq 0} \frac{t^n}{n!} G^n = \pi.
\]
So,
\[
	\e^\pi[f(X_t)] = \pi P_t f = \pi f = \sum_{x} \pi(x) f(x).
\]
The statement of the proposition now follows from taking $f$ to be the indicator of $y$:
\[
	f(x) = \begin{cases}
		1 & x=y\\
		0 & x\neq y.
	\end{cases}
\]
\end{proof}

\begin{fact}
	If $\pi$ is unique, then $P_t \to \pi$ as $t\to \infty$. That is, for all $f, x$, 
	\[
		\e^x[f(X_t)]\to \sum_y \pi(y) f(y) \quad \text{as }t\to \infty.
	\]
	This can also be phrased as saying that $X_t$ converges in distribution to $\pi$. 
\end{fact}





\end{document}
